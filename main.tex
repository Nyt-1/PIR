\documentclass[12pt]{article}

\usepackage{amsmath,amsthm,amssymb,amscd,epsfig,url}
\usepackage{fullpage}
\usepackage{tikz}
\usepackage{enumerate}
\usepackage{paralist}

\newtheorem{theorem}{Theorem}[section]
\newtheorem{lemma}[theorem]{Lemma}
\newtheorem{proposition}[theorem]{Proposition}
\newtheorem{corollary}[theorem]{Corollary}
\newtheorem{conjecture}[theorem]{Conjecture}
\newtheorem{definition}[theorem]{Definition}
\newtheorem{example}[theorem]{Example}
\newtheorem{remark}[theorem]{Remark}

\def\Z{{\mathbb Z}}
\def\Q{{\mathbb Q}}
\def\R{{\mathbb R}}
\def\N{{\mathbb N}}
\def\T{{\mathcal{C} T}}
\def\K{{\mathcal{C} K}}
\def\E{{\mathbb{E}}}
\newcommand{\hus}{\hfill\break}
\newcommand{\Div}{\mathrm{Div}}
\newcommand{\bull}{\textbullet\;\;}
\def\iso{{\operatorname{\ \approx \ }}}   % isomorvarphism
\def\im{{\operatorname{im}}}              % image
\def\Hom{{\operatorname{Hom}}}           % homomorvarphisms
\def\line{{\operatorname{line} \,}}       % line graph
\def\sd{{\operatorname{sd \,}}}           % first edge subdivision graph
\def\coker{{\operatorname{coker}}}        % cokernel
\def\lcm{{\operatorname{lcm}}}             % least common multiple
\newcommand{\gcdd}{\gamma}              
\def\del{{\partial}}                      % signed incidence matrix
\def\<{{\langle}} \def\>{{\rangle}}       % Inner product
\def\ZC{{Z^\#}}                           % cycle space
\setlength{\textheight}{9in}               
%\setlength{\textwidth}{6in}               % width of text
%\setlength{\oddsidemargin}{0in}           % odd page left margin
%\setlength{\evensidemargin}{0in}          % even page left margin 


\newcommand{\heading}[1]{\noindent\textbf{#1}}
\newcommand{\exercise}[1]{\heading{Exercise {#1}} }
\newcommand{\solution}{\heading{Solution.} }

\newcommand{\bits}{\{0,1\}}
\newcommand{\calS}{\mathcal{S}}
\newcommand{\calA}{\mathcal{A}}

\newcommand{\david}[1]{{\textbf{David says: {#1}}}}

\title{Some Notes about Grolmusz's Set Systems}

\begin{document}
\maketitle

\section{A Combinatorial Version of BBR Polynomials}

We will use the notation $[n] = \{1,2,\ldots,n\}$.

\begin{theorem}[Special Case of BBR]
  There exists a family of polynomials $Q_n \in \Z[X_1,\ldots,X_n]$,
  $n\in\N$, with
  the following properties:
  \begin{enumerate}
    \item $Q_n$ is multilinear, meaning that the exponent on any $X_i$ in
      any monomial is at most $1$.
    \item All coefficients of $Q_n$ are between $0$ and $5$.
    \item $Q_n(1,1,\ldots,1) \equiv 0 \pmod 6$.
    \item For all $x\in \bits^n\setminus \{1^n\}$, 
      $Q_n(x) \not\equiv 0 \pmod 6$.
    \item The degree of $Q_n$ is $O(\sqrt{n})$.
  \end{enumerate}
\end{theorem}
We note that given a polynomial satisfying only the last three properties, it is
easy to modify it so that it satisfies the first two as well.


Below we will use a correspondence between multilinear polynomials and families
of subsets. Here is an explicit example.
\begin{example}
  Let $Q(X_1,X_2,X_3) = 2X_1X_2 + X_1X_2X_3 + 4X_3$. We will first write $Q$ as
  a sum of monomials with coefficient $1$:
  \[
    Q(X_1,X_2,X_3) = X_1X_2 + X_1X_2 + X_1X_2X_3 + X_3 + X_3 + X_3 + X_3
  \]
  Now consider one of the monomials, say $X_1 X_2$. We can map this monomial to the
  set of its indices, which in this case is $\{1,2\}$. Applying this to all of
  the monomials, we get a multiset of subsets of $[n]$:
  \[
    \calS(Q) = \{ \{1,2\},  \{1,2\}, \{1,2,3\}, \{3\}, \{3\},\{3\},\{3\} \}.
  \]
  Note that $|\calS(Q)| = 7$, the sum of the coefficients of $Q$.
\end{example}

The follow corollary describes the combinatorial structure that is really needed
for Grolmusz's construction.
\begin{corollary}\label{cor:sets}
  For every $n\in\N$, there exists a multi-family $\calS$ of $D$ (possibly
  repeating) subsets of $[n]$ with the following properties:
  \begin{enumerate}
    \item $D \equiv 0 \pmod 6$
    \item For all $T\subseteq [n]$, $T\neq [n]$,
      \[
        \left|\{ S \in \calS :  S \subseteq T\}\right| \not\equiv 0
        \mod 6.
      \]
    \item $D = n^{O(\sqrt{n})}$.
    \item $\max_{S \in\calS} |S| = O(\sqrt{n})$.
  \end{enumerate}
\end{corollary}
\begin{proof}

  For a given $n$, take the polynomial $Q = Q_n$ from the theorem.  The family
  $\calS = \calS(Q)$ will satisfy the corollary, where $D = |\calS(Q)|$. 

  The first property follows since $Q(1,\ldots,1)$ is equal to the sum of the
  coefficients of $Q$, and so is $D = |\calS(Q)|$ (by the discussion at the end
  of the example).  By the theorem, this value is congruent to $0$ modulo $6$.

  To prove the second property, let $T\subseteq [n]$, $T\neq [n]$. Let $x_T \in
  \bits^n$ be the characteristic vector of $T$. Consider a monomial that
  corresponds to set $S \in \calS$.  Then this monomial evaluates to $1$
  on input $x_T$ if $S \subseteq T$, and $0$ otherwise. Since $Q(x_T)$ is the
  sum of the
  evaluations of all its monomials,
  \[
    Q(x_T) = \left|\{ S\in \calS :  S \subseteq T\}\right|.
  \]
  Now property (4) of $Q$ and the fact that $x_T \neq 1^n$  implies this value
  is not congruent $0$ modulo $6$, establishing the second property of $\calS$.

  We next consider the third property. Above we showed that $D$ is the
  sum of coefficients of $Q$. Since each coefficient is at most $5$, 
  we get that $D$ is at most $5$ times the number of monomials
  of $Q$. Let $d= \deg Q$. Since $d = O(\sqrt{n})$, we have
  $d \leq n/2$ for sufficiently large $n$. Then we have
  \[
    D \leq 5 \sum_{i=0}^{d} \binom{n}{i} \leq 5d \binom{n}{d}
    \leq 5d n^d = n^{O(\sqrt{n})}.
  \]

  The final property follows from the degree bound on $Q$.
\end{proof}

\section{Grolmusz's Construction}


\begin{theorem}
  For each $n\in\N$ there exists a family $\calA$ of $n^n$ subsets over a
  universe of size $n^{O(\sqrt{n})}$ with the following properties:
  \begin{enumerate}
    \item For all $A\in\calA$, $|A| \equiv 0 \pmod 6$.
    \item For all $A\neq A'\in \calA$, $|A \cap A'| \not\equiv 0 \pmod 6$.
  \end{enumerate}
\end{theorem}

The interesting part of this theorem is that if $6$ is replaced with a prime
then the largest possible $\calA$ has size $n^{O(\sqrt{n})}$.

\begin{proof}
  Apply the previous corollary to produce a multiset $\calS =
  \{S_1,\ldots,S_D\}$ satisfying the conditions of Corollary~\ref{cor:sets}.

  Fix a parameter $N\in\N$ to be discussed later (ultimately it is set to $n$).
  The universe will be
  \[
    U = \{  (j,z) : 1\leq j \leq D, z\in[N]^{|S_j|} \}.
  \]
  Note that $\calS$ was a multiset, meaning it allows repetitions, but
  $U$ is defined to be a \emph{set} since it refers to subscripts $j$, which
  are all unique, rather than sets $S_j$ which may repeat. 

  The family $\calA$ will have one member $A_x$ for each $x\in[N]^n$, where
  \[
    A_x = \{ (j,z) : \forall i\in S_j, z_i = x_i\}.
  \]
  We can think of $(j,z)$ as specifying a vector $v \in ([N]\cup\{\bot\})^n$,
  where the values of $z$ are placed at the positions pointed to by $S_j$,
  and the rest of the entries (not in in $S_j$) are set to $\bot$.
  Then $(j,z)\in A_x$ if and only if the non-$\bot$ entries of $v$ agree
  with the corresponding entries of $x$.

  We first calculate the size of the universe, in term of $N$. We have
  \[
    |U| = \sum_{j =1}^D N^{|S_j|} = 
    \sum_{j=1}^D N^{O(\sqrt{n})} =
     D N^{O(\sqrt{n})} = 
     n^{O(\sqrt{n})} N^{O(\sqrt{n})}.
  \]
  Recall that we'd like $U$ to be as small as possible, and $\calA$ to be as
  large as possible. From this bound we can see that setting $N=\theta(n)$
  results in no asymptotic ``cost'' in terms of the universe size, but
  larger settings would.

  Next, it is easy to see that $|\calA| = N^n$. Setting larger $N$ would be
  profitable here.

  Each $A_x \in \calA$ satisfies $|A_x| = D$, since for each $j \in [D]$,
  there is exactly one $z\in [N]^{|S_j|}$ that agrees with $x$ in the positions
  specified by $S_j$. Since $D \equiv 0 \pmod 6$, this first property follows.


  Now consider $A_x, A_y \in \calA$, $x \neq y$. We have
  \[
    A_x\cap A_y = \{ (j,z) : \forall i\in S_j, z_i = x_i = y_i\}.
  \]
  Let $T_{xy} = \{i \in [n] : x_i = y_i\}$. Note that $(j,z)$ can only be a member of
  $A_x \cap A_y$ if $x_i = y_i$ for all $i \in S_j$, so we must have $S_j \subseteq
  T_{xy}$.  In fact, for each $S_k\in\calS$ such that $S_k \subseteq T_{xy}$, there will be
  exactly one member $(k,z)\in A_x \cap A_y$, namely the member with the correct
  entries for $z$. Thus
  \[
    |A_x\cap A_y| = |\{ k \in [D] : S_k \subseteq T_{xy} \}|.
  \]
  By the second property of $\calS$, this value is not congruent to $0$ modulo
  $6$. (Note that $T_{xy} \neq [n]$.)

  As promised, if we set $N=n$ we get the claimed sizes of $U$ and $\calA$.
\end{proof}

\subsection{Interpreting the Bound and Explaining the Parameter Setting}

Let's informally recover the version the theorem where one fixes the universe
size so some $|U| = u$ and asks how large $|\calA|$ can be. In the theorem,
we got a sequence of constructions (one for each $n\in\N$) with
$u = n^{O(\sqrt{n})}$ and $|\calA| = n^n = u^{\Omega(\sqrt{n})}$. To get
a direct comparison we need to relate $\sqrt{n}$ to $u$. This follows
from
\[
  \log u = \log\left(n^{O(\sqrt{n})}\right)
  = O(\sqrt{n}\log n),
\]
and
\[
  \log\log u = \log\left(O(\sqrt{n}\log n)\right)
  = O(\log n + \log\log n) = O(\log n).
\]
Combining these we get (informally, at least)
\[
  \sqrt{n} = \Omega(\log u / \log \log u).
\]
Therefore we 
\[
  |\calA| = u^{\Omega\left(\frac{\log u}{\log \log u}\right)},
\]
which is the bound of Grolmusz.


Now we can go back and try to explain why $N$ was chosen to be $n$.  The proof
still works for any $N \geq 2$, but provides a different trade-off between $|U|$
and $|\calA|$. Taking $N < n$ wouldn't make sense because the universe size
would not shrink but $|\calA|$ would, which is the opposite of what we want.

We could, however, we take something like $N = n^n$. In this case $|U| =
n^{O(n^{1.5})}$ and $|\calA| = n^{n^2}$. 
But then we have
\[
  \log u = \log\left(n^{O(n)}\right) = O(n\log n),
\]
and
\[
  \log \log u = O(\log n),
\]
which implies
\[
  \sqrt{n} = \Omega(\sqrt{\log u/\log \log u}).
\]
Thus 
\[
  |\calA| = u^{\Omega\left(\sqrt{\frac{\log u}{\log \log u}}\right)},
\]
which is worse but not trivial.
\end{document}

